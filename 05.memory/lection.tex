\documentclass{beamer}

\usepackage[utf8]{inputenc}
\usepackage[russian]{babel}
\usepackage{tikz}
\usepackage{url}
\usepackage{xcolor}
\usepackage{listings}
\usepackage{verbatim}
\usepackage{ifthen}
\usepackage{bytefield}
\usepackage{wasysym}

\usetikzlibrary{positioning}
\usetikzlibrary{shapes}
\usetikzlibrary{arrows}
\usetikzlibrary{decorations.text}
\usetikzlibrary{chains}
\usetikzlibrary{calc}

\definecolor{yellowgreen}{HTML}{D0F000}
\definecolor{man}{HTML}{FF5E7C}
\definecolor{Agrey}{HTML}{AAAAAA}
\definecolor{Bgrey}{HTML}{BBBBBB}
\definecolor{Cgrey}{HTML}{CCCCCC}
\definecolor{Dgrey}{HTML}{DDDDDD}
\definecolor{Egrey}{HTML}{EEEEEE}

\definecolor{struct}{HTML}{999999}

% trace
\setbeamercolor{trace}{fg=black,bg=Cgrey}

% source
\setbeamercolor{source}{fg=black,bg=Cgrey}

% terminal
\setbeamercolor{terminal}{fg=white,bg=black}

% editor
\setbeamercolor{editor}{fg=black,bg=Cgrey}

% manual pages
\newcommand{\Man}[2]{
  \colorbox{white}{\color{man}#1(#2)}
}
\setbeamercolor{manref}{fg=black,bg=man}
\newcommand{\manlabel}[1]{
  \begin{beamercolorbox}[rounded=true,shadow=true,sep=0pt,colsep=0pt]{manref}
  \small{#1}
  \end{beamercolorbox}
}

% Source reference box
\setbeamercolor{srcref}{fg=black,bg=yellowgreen}
\newcommand{\srcref}[1]{
  \begin{beamercolorbox}[rounded=true,shadow=true,sep=0pt,colsep=0pt]{srcref}
  \tiny{Source code reference:}\footnotesize{ #1}
  \end{beamercolorbox}
}

% Shell command
\newcommand{\shellcmd}[1]{
  \begin{beamercolorbox}[rounded=true,shadow=true,sep=0pt,colsep=0pt]{terminal}
  \small{#1}
  \end{beamercolorbox}
}

% Source line
\newcommand{\srcline}[1]{
  \begin{beamercolorbox}[rounded=true,shadow=true,sep=0pt,colsep=0pt]{source}
  \small{#1}
  \end{beamercolorbox}
}

\tikzset { struct/.style = {
		draw, thick,
                rectangle split,
		rectangle split part fill={struct!50, white!50},
} }

%
% Footnotes for source/manual references.
%
\newcommand\ManRefSw{}
\newcommand\SrcRefSw{}
\makeatletter
\newcommand*\FootReferences[2]{
	\renewcommand\ManRefSw{#1}
	\renewcommand\SrcRefSw{#2}
	\beamer@calculateheadfoot
}
\makeatother
\setbeamertemplate{footline}{%
\ifthenelse{\NOT \equal{\ManRefSw}{}} {
	\begin{beamercolorbox}
	    [wd=\paperwidth,ht=2.25ex,dp=1ex,left]{manref}%
		\hspace*{1em}Manual page(s): \ManRefSw
	\end{beamercolorbox}
  }
  % else
  {}
\ifthenelse{\NOT \equal{\SrcRefSw}{}} {
	\begin{beamercolorbox}
	    [wd=\paperwidth,ht=2.25ex,dp=1ex,left]{srcref}%
		\hspace*{1em}Source code reference: \SrcRefSw
	\end{beamercolorbox}
  }
  % else
  {}
}


\title{Memory management in FreeBSD}

\begin{document}

\begin{frame}
\titlepage
\end{frame}


\begin{frame}
\frametitle{Process (static) address space}
\begin{figure}
\begin{tikzpicture}[start chain=going below, node distance=0mm]
    \tikzset {
	entry/.style={draw, thick, on chain, text width=.25\paperwidth,
		      align=center},
	large/.style={entry, minimum height=.15\paperheight},
	noentry/.style={entry, fill=gray, shading=axis, shading angle=45},
	nolarge/.style={large, fill=gray, shading=axis, shading angle=45},
    }
    \node [name=kernel, nolarge] { kernel };
    \node [name=argv, entry] { argv, envp };
    \node [name=stack, large] { stack };
    \node [name=unmap, nolarge] { not mapped memory };
    \node [name=heap, large] { heap };
    \node [name=data, entry] { initialized data };
    \node [name=text, entry] { program text };
    \node [name=null, noentry] { };

    \draw [<-] (kernel.north west) --
	node [below] {\scriptsize{0xff \ldots ff}} +(-1.5cm, 0);
    \draw [<-] (null.south west) --
	node [above] {\scriptsize{0x00}} +(-1.5cm, 0);
\end{tikzpicture}
\end{figure}
\end{frame}

\FootReferences{procstat(1)}{}
\begin{frame}
\frametitle{Process (static) address space}
Try this out:
\shellcmd{%
\# /rescue/cat \&\\
\lbrack1\rbrack 48141\\
\# procstat -v \$!\\
\begin{tabular}{rrrrrrrrrrr}
  PID &            START &              END & PRT  & RES & PRES & REF & SHD &   FL & TP & PATH \\
48141 &         0x400000 &         0x96d000 & r-x  & 677 &    0 &   4 &   2 & CN-- & vn & /rescue/chio \\
48141 &         0xb6d000 &         0xb94000 & rw-  &  38 &    0 &   1 &   0 & C--- & vn & /rescue/chio \\
48141 &         0xb94000 &         0xdc8000 & rw-  &  21 &    0 &   1 &   0 & ---- & df & \\
48141 &      0x800c00000 &      0x801400000 & rw-  &  81 &    0 &   1 &   0 & ---- & df & \\
48141 &   0x7fffffbfe000 &   0x7fffffbff000 & ---  &   0 &    0 &   0 &   0 & ---- & -- & \\
48141 &   0x7ffffffdf000 &   0x7ffffffff000 & rw-  &   4 &    0 &   1 &   0 & ---D & df & \\
48141 &   0x7ffffffff000 &   0x800000000000 & r-x  &   0 &    0 &  67 &   0 & ---- & ph & \\
\end{tabular}
}
\end{frame}


\FootReferences{}{}
\begin{frame}
\frametitle{Stack grows implicitly}
\begin{columns}
\begin{column}{.4\paperwidth}
  \begin{figure}
  \begin{tikzpicture}[start chain=going below, node distance=0mm]
    \tikzset {
	entry/.style={draw, thick, on chain, text width=.25\paperwidth,
		      align=center},
	large/.style={entry, minimum height=.15\paperheight},
	noentry/.style={entry, fill=gray, shading=axis, shading angle=45},
	nolarge/.style={large, fill=gray, shading=axis, shading angle=45},
    }
    \node [name=argv, entry] { argv, envp };
    \node [name=stack, large] { stack };
    \node [name=unmap, nolarge] { not mapped memory };
    \node [name=heap, large] { heap };
    \node [name=data, entry] { initialized data };
    \node [name=text, entry] { program text };
    \node [name=null, noentry] { };

    \draw (stack.south west) ++(0cm, -2mm) [<-] --
	node [above] {\scriptsize{deref!}} +(-1.5cm, 0);
  \end{tikzpicture}
  \end{figure}
\end{column}
\begin{column}{.2\paperwidth}
  \begin{tikzpicture}
    \draw [->, thick] (0,0) -- node [above] {\small{page fault}}
	+(.2\paperwidth,0);
  \end{tikzpicture}
\end{column}
\begin{column}{.4\paperwidth}
  \begin{figure}
  \begin{tikzpicture}[start chain=going below, node distance=0mm]
    \tikzset {
	entry/.style={draw, thick, on chain, text width=.25\paperwidth,
		      align=center},
	large/.style={entry, minimum height=.15\paperheight},
	noentry/.style={entry, fill=gray, shading=axis, shading angle=45},
	nolarge/.style={large, fill=gray, shading=axis, shading angle=45},
    }
    \node [name=argv, entry] { argv, envp };
    \node [name=stack, large, minimum height=.20\paperheight] { stack };
    \node [name=unmap, nolarge, minimum height=.10\paperheight]
	{ not mapped memory };
    \node [name=heap, large] { heap };
    \node [name=data, entry] { initialized data };
    \node [name=text, entry] { program text };
    \node [name=null, noentry] { };

    \draw [thick] (stack.center) ++(-1cm,0) node (mark1) {}
	[->] -- (mark1 |- stack.south);
    \draw [thick] (stack.center) ++(1cm,0) node (mark2) {}
	[->] -- (mark2 |- stack.south);
  \end{tikzpicture}
  \end{figure}
\end{column}
\end{columns}
\end{frame}


\FootReferences{sbrk(2)}{}
\begin{frame}
\frametitle{Heap grows explicitly (used to grow \smiley )}
\begin{columns}
\begin{column}{.4\paperwidth}
  \begin{figure}
  \begin{tikzpicture}[start chain=going below, node distance=0mm]
    \tikzset {
	entry/.style={draw, thick, on chain, text width=.25\paperwidth,
		      align=center},
	large/.style={entry, minimum height=.15\paperheight},
	noentry/.style={entry, fill=gray, shading=axis, shading angle=45},
	nolarge/.style={large, fill=gray, shading=axis, shading angle=45},
    }
    \node [name=argv, entry] { argv, envp };
    \node [name=stack, large] { stack };
    \node [name=unmap, nolarge] { not mapped memory };
    \node [name=heap, large] { heap };
    \node [name=data, entry] { initialized data };
    \node [name=text, entry] { program text };
    \node [name=null, noentry] { };
  \end{tikzpicture}
  \end{figure}
\end{column}
\begin{column}{.2\paperwidth}
  \begin{tikzpicture}
    \draw [->, thick] (0,0) -- node [above] {\small{sbrk(2)}}
	+(.2\paperwidth,0);
  \end{tikzpicture}
\end{column}
\begin{column}{.4\paperwidth}
  \begin{figure}
  \begin{tikzpicture}[start chain=going below, node distance=0mm]
    \tikzset {
	entry/.style={draw, thick, on chain, text width=.25\paperwidth,
		      align=center},
	large/.style={entry, minimum height=.15\paperheight},
	noentry/.style={entry, fill=gray, shading=axis, shading angle=45},
	nolarge/.style={large, fill=gray, shading=axis, shading angle=45},
    }
    \node [name=argv, entry] { argv, envp };
    \node [name=stack, large] { stack };
    \node [name=unmap, nolarge, minimum height=.10\paperheight]
	{ not mapped memory };
    \node [name=heap, large, minimum height=.20\paperheight] { heap };
    \node [name=data, entry] { initialized data };
    \node [name=text, entry] { program text };
    \node [name=null, noentry] { };

    \draw [thick] (heap.center) ++(-1cm,0) node (mark1) {}
	[->] -- (mark1 |- heap.north);
    \draw [thick] (heap.center) ++(1cm,0) node (mark2) {}
	[->] -- (mark2 |- heap.north);
  \end{tikzpicture}
  \end{figure}
\end{column}
\end{columns}
\end{frame}


\FootReferences{mmap(2)}{}
\begin{frame}
\frametitle{Mapping more on heap}
\begin{columns}
\begin{column}{.4\paperwidth}
  \begin{figure}
  \begin{tikzpicture}[start chain=going below, node distance=0mm]
    \tikzset {
	entry/.style={draw, thick, on chain, text width=.25\paperwidth,
		      align=center},
	large/.style={entry, minimum height=.15\paperheight},
	noentry/.style={entry, fill=gray, shading=axis, shading angle=45},
	nolarge/.style={large, fill=gray, shading=axis, shading angle=45},
    }
    \node [name=argv, entry] { argv, envp };
    \node [name=stack, large] { stack };
    \node [name=unmap, nolarge] { not mapped memory };
    \node [name=heap, large] { heap };
    \node [name=data, entry] { initialized data };
    \node [name=text, entry] { program text };
    \node [name=null, noentry] { };
  \end{tikzpicture}
  \end{figure}
\end{column}
\begin{column}{.2\paperwidth}
  \begin{tikzpicture}
    \draw [->, thick] (0,0) -- node [above] {\small{mmap(2)}}
	+(.2\paperwidth,0);
  \end{tikzpicture}
\end{column}
\begin{column}{.4\paperwidth}
  \begin{figure}
  \begin{tikzpicture}[start chain=going below, node distance=0mm]
    \tikzset {
	entry/.style={draw, thick, on chain, text width=.25\paperwidth,
		      align=center},
	large/.style={entry, minimum height=.15\paperheight},
	noentry/.style={entry, fill=gray, shading=axis, shading angle=45},
	nolarge/.style={large, fill=gray, shading=axis, shading angle=45},
    }
    \node [name=argv, entry] { argv, envp };
    \node [name=stack, large] { stack };
    \node [name=unmap1, nolarge, minimum height=.05\paperheight]
	{ not mapped };
    \node [name=mmap, large, minimum height=.05\paperheight]
	{ more heap };
    \node [name=unmap2, nolarge, minimum height=.05\paperheight]
	{ not mapped };
    \node [name=heap, large] { heap };
    \node [name=data, entry] { initialized data };
    \node [name=text, entry] { program text };
    \node [name=null, noentry] { };
  \end{tikzpicture}
  \end{figure}
\end{column}
\end{columns}
\end{frame}


\FootReferences{mmap(2), malloc(3), rtld(1)}{}
\begin{frame}
\frametitle{Modern process memory map}
\begin{columns}
\begin{column}{.5\paperwidth}
  \begin{figure}
  \begin{tikzpicture}[start chain=going below, node distance=0mm]
    \tikzset {
	entry/.style={draw, thick, on chain, text width=.25\paperwidth,
		      align=center},
	large/.style={entry, minimum height=.1\paperheight},
	noentry/.style={entry, fill=gray, shading=axis, shading angle=45},
	nolarge/.style={large, fill=gray, shading=axis, shading angle=45},
    }
    \node [name=argv, entry] { argv, envp };
    \node [name=stack, entry] { stack };
    \node [name=unmap1, nolarge] { not mapped };
    \node [name=mmap, entry] { malloc arena };
    \node [name=unmap2, nolarge] { not mapped };
    \node [name=lib, entry] { library };
    \node [name=unmap3, nolarge] { not mapped };
    \node [name=heap, entry] { ld-elf };
    \node [name=data, entry] { initialized data };
    \node [name=text, entry] { program text };
    \node [name=null, noentry] { };
  \end{tikzpicture}
  \end{figure}
\end{column}
\begin{column}{.5\paperwidth}
\begin{itemize}
  \item{program itself via mmap(2)}
  \item{dynamic libraries via mmap(2)}
  \item{malloc(3) arenas via mmap(2)}
\end{itemize}
\onslide<2-> {
    Try this out:
    \shellcmd{%
    \# /bin/cat \&\\
    \# procstat -v \$!\\
    }
    Or this:
    \shellcmd{%
    \# procstat -v \$\$\\
    }
}
\end{column}
\end{columns}
\end{frame}


\FootReferences{}{}
\begin{frame}
\frametitle{VM space}
\begin{figure}
\begin{tikzpicture}[start chain=going below, node distance=0mm]
    \tikzset {
	entry/.style={draw, thick, on chain, text width=.25\paperwidth,
		      align=center},
	large/.style={entry, minimum height=.1\paperheight},
	noentry/.style={entry, fill=gray, shading=axis, shading angle=45},
	nolarge/.style={large, fill=gray, shading=axis, shading angle=45},
    }
    \node [name=argv, entry] { argv, envp };
    \node [name=stack, entry] { stack };
    \node [name=unmap1, nolarge] { not mapped };
    \node [name=mmap, entry] { malloc arena };
    \node [name=unmap2, nolarge] { not mapped };
    \node [name=lib, entry] { library };
    \node [name=unmap3, nolarge] { not mapped };
    \node [name=heap, entry] { ld-elf };
    \node [name=data, entry] { initialized data };
    \node [name=text, entry] { program text };
    \node [name=null, noentry] { };

    \node [name=vmspace, draw, thick, color=red, ellipse,
	  minimum width=.35\paperwidth,
	  minimum height=.8\paperheight] at (lib.north) {};
    \node [ellipse callout, draw, color=red,
	   callout absolute pointer={(node cs:name=vmspace, angle=40)},
	   node distance=5mm, right=of stack] { vm\_map };

    \node [name=vmentry, draw, thick, color=red, ellipse,
	  minimum width=.30\paperwidth,
	  minimum height=2em] at (lib.center) {};
    \node [ellipse callout, draw, color=red,
	   callout absolute pointer={(node cs:name=vmentry, angle=2)},
	   node distance=7mm, right=of unmap2] { vm\_map\_entry };
\end{tikzpicture}
\end{figure}
\end{frame}


\FootReferences{}{sys/vm/vm\_map.h}
\begin{frame}
\frametitle{Kernel representation of VM space}
\begin{figure}
\small\begin{tikzpicture}
  \node [name=vmspace, struct, rectangle split parts=3] {
	\textbf{struct vmspace}
	\nodepart{two} struct vm\_map
	\nodepart{three} struct vm\_pmap
  };
  \node [name=vmmap, struct, right=of vmspace, rectangle split parts=4] {
	\textbf{struct vm\_map}
	\nodepart{two} struct vm\_map\_entry header
	\nodepart{three} struct vm\_map\_entry \*tree
	\nodepart{four} \ldots
  };
  \node [name=entry1, struct, below=of vmspace, rectangle split parts=5] {
	\textbf{struct vm\_map\_entry}
	\nodepart{two} struct vm\_map\_entry *next
	\nodepart{three} vm\_offset\_t start
	\nodepart{four} vm\_offset\_t end
	\nodepart{five} \ldots
  };
  \node [name=entry2, struct, right=of entry1, rectangle split parts=5] {
	\textbf{struct vm\_map\_entry}
	\nodepart{two} struct vm\_map\_entry *next
	\nodepart{three} vm\_offset\_t start
	\nodepart{four} vm\_offset\_t end
	\nodepart{five} \ldots
  };

  \node [name=mark, node distance=5mm, above left=of entry1] {};
  \draw [->, thick, rounded corners] (vmmap.two east) -- ++(5mm,0)
	|- (mark.center) |- (entry1.one west);

  \draw [->, thick] (entry1.two east) to [out=0, in=180] (entry2.one west);
  \draw [->, thick, rounded corners] (entry2.two east) --
	++(5mm,0) -- ++(0,-1cm);
\end{tikzpicture}
\end{figure}
\end{frame}


\FootReferences{}{sys/vm/vm\_map.h, sys/vm/vm\_object.h}
\begin{frame}
\frametitle{A VM map entry is backed by an object}
\begin{figure}
\begin{tikzpicture}
  \tikzset {
	page/.style={draw, thick, node distance=3mm},
  }
  \node [name=entry, struct, rectangle split parts=4] {
	\textbf{struct vm\_map\_entry}
	\nodepart{two} \ldots
	\nodepart{three} struct vm\_object *object
	\nodepart{four} \ldots
  };
  \node [name=object, struct, right=of entry, rectangle split parts=3] {
	\textbf{struct vm\_object}
	\nodepart{two} struct vm\_radix head
	\nodepart{three} union *pager
  };

  \draw [->, thick] (entry.three east) to [out=0, in=180] (object.one west);

  \node [name=page1, page, below=of object] { vm\_page };
  \draw [->, thick, rounded corners] (object.two east) -- ++(1cm,0)
	|- (page1.east);

  \node [name=page2, page, below right=of page1] { vm\_page };
  \node [name=page3, page, below left=of page1] { vm\_page };
  \draw [->,thick] (page1.south) to [out=290, in=180] (page2.west);
  \draw [->,thick] (page1.south) to [out=250, in=0] (page3.east);

  \node [name=page4, page, below right=of page3] { vm\_page };
  \node [name=page5, page, below left=of page3] { vm\_page };
  \draw [->,thick] (page3.south) to [out=290, in=180] (page4.west);
  \draw [->,thick] (page3.south) to [out=250, in=0] (page5.east);

  \node [name=page6, page, below right=of page5] { vm\_page };
  \node [name=page7, page, below left=of page5] { vm\_page };
  \draw [->,thick] (page5.south) to [out=290, in=180] (page6.west);
  \draw [->,thick] (page5.south) to [out=250, in=0] (page7.east);

  \node [name=page8, page, below right=of page4] { vm\_page };
  \draw [->,thick] (page4.south) to [out=290, in=180] (page8.west);

\end{tikzpicture}
\end{figure}
\end{frame}


\end{document}
