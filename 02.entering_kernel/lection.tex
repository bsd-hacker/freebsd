\documentclass{beamer}

\usepackage[utf8]{inputenc}
\usepackage[russian]{babel}
\usepackage{tikz}
\usepackage{adjustbox}
\usepackage{url}
\usepackage{array}
\usepackage{xcolor}
\usepackage{listings}
\usepackage{verbatim}
\usepackage{ifthen}

\usetikzlibrary{positioning}
\usetikzlibrary{shapes.arrows}
\usetikzlibrary{decorations.pathmorphing}

\definecolor{yellowgreen}{HTML}{D0F000}
\definecolor{man}{HTML}{FF5E7C}
\definecolor{Agrey}{HTML}{AAAAAA}
\definecolor{Bgrey}{HTML}{BBBBBB}
\definecolor{Cgrey}{HTML}{CCCCCC}
\definecolor{Dgrey}{HTML}{DDDDDD}
\definecolor{Egrey}{HTML}{EEEEEE}

\definecolor{struct}{HTML}{999999}

% trace
\setbeamercolor{trace}{fg=black,bg=Cgrey}

% source
\setbeamercolor{source}{fg=black,bg=Cgrey}

% terminal
\setbeamercolor{terminal}{fg=white,bg=black}

% editor
\setbeamercolor{editor}{fg=black,bg=Cgrey}

% manual pages
\newcommand{\Man}[2]{
  \colorbox{white}{\color{man}#1(#2)}
}
\setbeamercolor{manref}{fg=black,bg=man}
\newcommand{\manlabel}[1]{
  \begin{beamercolorbox}[rounded=true,shadow=true,sep=0pt,colsep=0pt]{manref}
  \small{#1}
  \end{beamercolorbox}
}

% Source reference box
\setbeamercolor{srcref}{fg=black,bg=yellowgreen}
\newcommand{\srcref}[1]{
  \begin{beamercolorbox}[rounded=true,shadow=true,sep=0pt,colsep=0pt]{srcref}
  \tiny{Source code reference:}\footnotesize{ #1}
  \end{beamercolorbox}
}

% Shell command
\newcommand{\shellcmd}[1]{
  \begin{beamercolorbox}[rounded=true,shadow=true,sep=0pt,colsep=0pt]{terminal}
  \small{#1}
  \end{beamercolorbox}
}

% Source line
\newcommand{\srcline}[1]{
  \begin{beamercolorbox}[rounded=true,shadow=true,sep=0pt,colsep=0pt]{source}
  \small{#1}
  \end{beamercolorbox}
}

\tikzset { struct/.style = {
		draw, thick,
                rectangle split,
		rectangle split part fill={struct!50, white!50},
} }

%
% Footnotes for source/manual references.
%
\newcommand\ManRefSw{}
\newcommand\SrcRefSw{}
\makeatletter
\newcommand*\FootReferences[2]{
	\renewcommand\ManRefSw{#1}
	\renewcommand\SrcRefSw{#2}
	\beamer@calculateheadfoot
}
\makeatother
\setbeamertemplate{footline}{%
\ifthenelse{\NOT \equal{\ManRefSw}{}} {
	\begin{beamercolorbox}
	    [wd=\paperwidth,ht=2.25ex,dp=1ex,left]{manref}%
		\hspace*{1em}Manual page(s): \ManRefSw
	\end{beamercolorbox}
  }
  % else
  {}
\ifthenelse{\NOT \equal{\SrcRefSw}{}} {
	\begin{beamercolorbox}
	    [wd=\paperwidth,ht=2.25ex,dp=1ex,left]{srcref}%
		\hspace*{1em}Source code reference: \SrcRefSw
	\end{beamercolorbox}
  }
  % else
  {}
}


\title{Entering and exiting the kernel:
interrupts, traps, system calls}

\begin{document}

\begin{frame}
\titlepage
\end{frame}


\begin{frame}
\frametitle<1>{CPU is a Turing maching}
\frametitle<2>{Interrupt}
\frametitle<3>{Saving entire context}
\begin{tikzpicture}[node distance=2mm, very thick]
  \draw [anchor=north] (2, 0) node[name=cpu, draw, rounded corners=3mm,
	minimum height=5cm, minimum width=4cm]{};
  \node [name=cputext, below=of cpu.north]{CPU};
  \node [name=ip, below=of cputext, draw, text width=3cm, align=center]
	{ instruction pointer (program counter) };
  \node [name=sp, below=of ip, draw, text width=3cm, align=center]
	{ stack pointer };
  \node [name=reg0, below=of sp.south west, anchor=north west, draw,
	align=center, minimum width = 1.5 cm]
	{ reg0 };
  \node [name=reg1, below=of sp.south east, anchor=north east, draw,
	align=center, minimum width = 1.5 cm]
	{ reg1 };
  \node [name=reg2, below=of reg0.south west, anchor=north west, draw,
	align=center, minimum width = 1.5 cm]
	{ reg2 };
  \node [name=reg3, below=of reg1.south east, anchor=north east, draw,
	align=center, minimum width = 1.5 cm]
	{ reg3 };

  \draw [anchor=north] (8, 0) node[name=mem, draw, rounded corners=3mm,
	minimum height=7cm, minimum width=7cm]{};
  \node [name=memtext, below=of mem.north]{Memory};

  % Stack
  \node [name=stack0, above right=1cm and 0.5cm of mem.west, draw,
	outer sep=0pt, minimum height=1cm, minimum width=2cm] { };
  \node [name=stack1, node distance = 0mm, below=of stack0, draw, outer sep=0pt,
	minimum height=1cm, minimum width=2cm] { 0 };
  \node [name=stack2, node distance = 0mm, below=of stack1, draw, outer sep=0pt,
	minimum height=1cm, minimum width=2cm] { 2 };
  \node [name=stack3, node distance = 0mm, below=of stack2, draw, outer sep=0pt,
	minimum height=1cm, minimum width=2cm] { 1 };
  \node [name=stacktext, node distance = 0mm, above=of stack0] { stack };
  \draw [loosely dashed] (stack0.north east) -- +(0,1);
  \draw [loosely dashed] (stack0.north west) -- +(0,1);
  \draw [loosely dashed] (stack3.south east) -- +(0,-1);
  \draw [loosely dashed] (stack3.south west) -- +(0,-1);

  % Code
  \node [name=code0, above left=1cm and 0.5cm of mem.east, draw, outer sep=0pt,
	minimum height=1cm, minimum width=3cm] { and reg1,reg2 };
  \node [name=code1, node distance = 0mm, below=of code0, draw, outer sep=0pt,
	minimum height=1cm, minimum width=3cm] { test reg1, 0xf };
  \node [name=code2, node distance = 0mm, below=of code1, draw, outer sep=0pt,
	minimum height=1cm, minimum width=3cm] { jnz +0x10 };
  \draw [loosely dashed] (code0.north east) -- +(0,1);
  \draw [loosely dashed] (code0.north west) -- +(0,1);
  \draw [loosely dashed] (code2.south east) -- +(0,-1);
  \draw [loosely dashed] (code2.south west) -- +(0,-1);

  \onslide<1> {
    \note<1> {
	- CPU is like a Turing machine.\\
	- Explain what IP or program counter means.\\
	- Why pure TM doesn't represent what we want from a computer?
	  We don't want computer to endlessly run a calculation, we want
	  it to be interactive. User input should interrupt it. And not
	  only user input.\\
    }
    \draw [->] (sp.east) to [out=15, in=165] (stack1.west);
    \draw [->] (ip.east) to [out=-15, in=165] (code0.west);
  }

  \onslide<2> {
    \note<2> {
	- So, interruption of computing is important.\\
	- Interrupt should be transparent for current execution. How
	  can this be achieved?\\
    }
    \draw [->, dashed] (sp.east) to [out=15, in=165] (stack1.west);
    \draw [->, dashed] (ip.east) to [out=-15, in=165] (code0.west);
    \draw [->, color=red] (sp.east) to [out=15, in=165] (stack0.west);
    \node [name=addr0, draw, circle, color=red, minimum width=5mm]
  	at (code0.west) {};
    \draw [->, color=red] (addr0.south west) to [out=200, in=300]
	(stack0.center);
  
    % Intr
    \node [name=intr0, node distance = 0mm, below=1cm of code2, draw,
  	outer sep=0pt, minimum height=1cm, minimum width=3cm] { push reg1 };
    \node [node distance=0, above=of intr0] { \footnotesize interrupt handler };
    \draw [->, color=red] (ip.east) to [out=30,in=165] (intr0.west);
  }

  \onslide<3> {
    \note<3> {
	- Saving entire context is important in some cases.\\
	- And what if on return interrupt handler provides another context?\\
	- Handling multiple contexts == multitasking.\\
    }
    \node [name=ctx, draw, circle, color=red, minimum width=4cm]
	at (cpu.center) {};
    \draw [->, color=red] (ctx.north east) to [out=30, in=140]
	(stacktext.north);
  }
\end{tikzpicture}
\onslide<3> {
  \srcref{sys/i386/i386/swtch.s, sys/amd64/amd64/cpu\_switch.S
}
}
\end{frame}


\begin{frame}
\frametitle{Interrupts}
\note {
	- Let's see what kind of interrupts we got.\\
}
\begin{itemize}
  \onslide<1->{
    \note<1> {
	- Okay, now we can do multitasking and interactivity.\\
    }
    \item{Hardware interrupt (asynchoronous, involuntary)
      \begin{itemize}
	\item I/O ready
	\item I/O done
	\item timer tick
      \end{itemize}
    }
  }
  \onslide<2->{
    \note<2> {
	- We can handle (and "fix") traps, and even implement paging.\\
    }
    \item{Trap (synchoronous, involuntary)
      \begin{itemize}
	\item page fault
	\item general protection fault
      \end{itemize}
    }
  }
  \onslide<3->{
    \note<3> {
	- Via interrupts we can provide a library of functions, what
	  actually MS DOS did via INT 0x21. Actually many software of
	  the DOS era were setting a library on a particular interrupt.\\
    }
    \item{Software interrupt (synchoronous, voluntary)\\
	On x86 the ``int'' instruction
    }
  }
\end{itemize}
\end{frame}


\begin{frame}
\frametitle{Protection}
\note<1> {
	- What isn't enough for a true OS is protection of tasks against
	  each other.\\
	- We need individual address space for each, and only a supervisor
	  (the kernel) that can set 'em up.\\
	- We need to forbid certain priveleged instructions to anyone
	  except supervisor.\\
	- This is what 80386 processor gave us.\\
	- Now the library of functions lives in ring0, and is called
	  kernel.\\
}
\begin{tikzpicture}[very thick]
  \node[name=ring3, draw, rounded corners=3mm,
	minimum width=11cm, minimum height=3cm] {};
  \node[name=ring0, node distance=1.4cm, below=of ring3, draw,
	rounded corners=3mm, minimum width=11cm, minimum height=3cm] {};
  \node[name=int, single arrow, node distance=2mm, anchor=north,
	below right=1mm and 2 cm of ring3.south west, draw,
	minimum width=4em, minimum height=3em, shape border rotate=270]
	{ int };
  \node[name=trap, single arrow, right=of int, draw,
	minimum width=4em, minimum height=3em, shape border rotate=270]
	{ \emph{trap} };
  \node[name=iret, single arrow, node distance=2mm, anchor=north,
	above left=1mm and 2 cm of ring0.north east, draw,
	minimum width=4em, minimum height=3em, shape border rotate=90]
	{ iret };

  \onslide<1>{
    \node at (ring3.center) { \LARGE{ring 3} };
    \node at (ring0.center) { \LARGE{ring 0} };
  }

  \onslide<2>{
    \node at (ring3.center) { \color{red}\LARGE{userland} };
    \node at (ring0.center) { \color{red}\LARGE{kernel} };
  }
\end{tikzpicture}
\end{frame}


\begin{frame}
\frametitle{System call}
\note<1> {
	- Let's make an agreement that certain software interrupt has an
	  ABI that would provide a gate into kernel.\\
	- Using variety of syscall numbers we can implement all needed
	  services that kernel offers to userland.\\
	- The int 0x80 is already a history. Modern hardware provide
	  special much more efficient instruction designed for syscall.
	  There also were "call gates".\\
}
\onslide<1->{
  \begin{beamercolorbox}[rounded=true,shadow=true,sep=0pt,ht=2.9cm]{trace}
    Application
    \begin{itemize}
      \item Puts syscall number into a register
      \item Puts arguments into registers or into memory
      \item Triggers interrupt (i386: \emph{int 0x80}, amd64: \emph{syscall})
      \onslide<3->{
	\item Resumes execution, checks return value, reads data, etc.
      }
    \end{itemize}
    \end{beamercolorbox}
}
\onslide<2->{
  \begin{beamercolorbox}[rounded=true,shadow=true,sep=0pt,ht=3.4cm]{trace}
    Kernel
    \begin{itemize}
      \item Reads syscall number from a register
      \item Reads arguments from registers or into memory
      \item Does the processing
      \item Puts return value into a register, copies data
      \item Returns from interrupt (i386: \emph{iret}, amd64: \emph{sysret})
    \end{itemize}
    \end{beamercolorbox}
}
\srcref{lib/libc/i386/sys/syscall.S, sys/i386/i386/trap.c,
lib/libc/amd64/sys/SYS.h, sys/amd64/amd64/trap.c}
\end{frame}


\begin{frame}[fragile]
\frametitle{System call example}
\note<1> {
	- Let's trace an application ping(8). It sends data to network,
	  so we expect it to use send(2). Our outbound interface is Ethernet,
	  so we expect ether\_output() to be in action.\\
	- Arm gdb \& dtrace appropriately!\\
	- Note that data is taken from dtrace and gdb, but order is
	  reversed!\\
	- Note that we expected ICMP but got UDP on first send() due to
	  resolving, but that still works for us as an example! Also, it
	  demonstrates that library calls can use syscalls implicitly.\\
	- Explain user stack and kernel stack.\\
	- Let's trace recvfrom(), too.\\
}
\shellcmd{\# ping www.ru}
\begin{scriptsize}
\begin{onlyenv}<1>
  \begin{beamercolorbox}[rounded=true,shadow=true,sep=0pt,ht=3.5cm]{trace}
	\begin{verbatim}
	#9  main () at ping.c:563
	#8  gethostbyname2 () from /lib/libc.so.7
	#7  gethostbyname_r () from /lib/libc.so.7
	#6  nsdispatch () from /lib/libc.so.7
	#5  __dns_getanswer () from /lib/libc.so.7
	#4  __res_nsearch () from /lib/libc.so.7
	#3  __res_nquerydomain () from /lib/libc.so.7
	#2  __res_nquery () from /lib/libc.so.7
	#1  __res_nsend () from /lib/libc.so.7
	#0  send () from /lib/libc.so.7
	\end{verbatim}
  \end{beamercolorbox}
  \begin{beamercolorbox}[rounded=true,shadow=true,sep=0pt,ht=3cm]{trace}
	\begin{verbatim}
	kernel`amd64_syscall+0x357
	kernel`sys_sendto+0x4d
	kernel`sendit+0x116
	kernel`kern_sendit+0x224
	kernel`sosend_dgram+0x2f2
	kernel`udp_send+0x855
	kernel`ip_output+0xf0a
	ether_output:entry 
	\end{verbatim}
  \end{beamercolorbox}
\end{onlyenv}
\begin{onlyenv}<2>
  \begin{beamercolorbox}[rounded=true,shadow=true,sep=0pt,ht=3.5cm]{trace}
	\begin{verbatim}
	#9  main (argc=<value optimized out>, argv=0x0) at ping.c:563
	#8  gethostbyname2 () from /lib/libc.so.7
	#7  gethostbyname_r () from /lib/libc.so.7
	#6  nsdispatch () from /lib/libc.so.7
	#5  __dns_getanswer () from /lib/libc.so.7
	#4  __res_nsearch () from /lib/libc.so.7
	#3  __res_nquerydomain () from /lib/libc.so.7
	#2  __res_nquery () from /lib/libc.so.7
	#1  __res_nsend () from /lib/libc.so.7
	#0  recvfrom () from /lib/libc.so.7
	\end{verbatim}
  \end{beamercolorbox}
  \begin{beamercolorbox}[rounded=true,shadow=true,sep=0pt,ht=2cm]{trace}
	\begin{verbatim}
	kernel`amd64_syscall+0x357
	kernel`sys_recvfrom+0x86
	kernel`kern_recvit+0x22f
	kernel`soreceive_dgram+0x45d
	uiomove:entry 
	\end{verbatim}
  \end{beamercolorbox}
\end{onlyenv}
\end{scriptsize}
\end{frame}


\begin{frame}
\frametitle{Hardware interrupts and system calls}
\note<1> {
	- We can enter kernel from a hardware interrupt, or from
	  a software interrupt. The former is usually a driver entry,
	  and the latter is a syscall.\\
	- Drivers can have pure interrupt handlers, but more often
	  have ISRs.\\
	- Kernel is somewhat divided into two parts... Well, it is
	  really not. :) But some threads are always calling "down",
	  and some are calling "up". And they are tightly interconnected.\\
	- They usually synchronise at a some data structure, a generic
	  buffer.\\
}
\begin{tikzpicture}[very thick]
  \node[name=user, draw, rounded corners=3mm,
	minimum width=11cm, minimum height=2cm] {};
  \node [name=usertext] at (user.center)
	{ \LARGE{userland} };
  \node [name=ping, right=of user.west, draw, rounded corners=1mm]
	{ ping };
  \node[name=kernel, node distance=2mm, below=of user, draw,
	rounded corners=3mm, minimum width=11cm, minimum height=6cm] {};
  \draw [loosely dashed,decorate,decoration={snake,amplitude=0.25cm,segment length=2cm}] (kernel.west) -- (kernel.east);
  \node [name=top, node distance=2mm, below right=of kernel.north]
	{ \LARGE{kernel (top half)} };
  \node [name=bottom, node distance=2mm, above right=of kernel.south]
	{ \LARGE{kernel (bottom half)} };
  \node [name=buf, left=of kernel.center, draw, fill=white, circle]
	{ buffer };
  \node [name=ether, above right=2mm and 1cm of kernel.south west,
	draw, rounded corners=1mm]
	{ Ethernet driver };
  \draw [->] (ether.north) to [out=110, in=250]
	node [midway, sloped, below] {\footnotesize interrupt}
	(buf.south west);
  \draw [->] (ping.south) to [out=250, in=110]
	node [midway, sloped, above] {\footnotesize send()}
	(buf.north west);
  \draw [->] (ping.south) to [out=290, in=40]
	node [midway, sloped, above] {\footnotesize recv()}
	(buf.north east);

  \onslide<2>{
    \note<2> {
	- A thread coming from userland can eventually reach the very
	  bottom of a driver.\\
	- A thread coming from a driver can't pass kernel to userland
	  boundary.\\
    }
    \draw [->] (ping.south) to [out=230, in=120]
	node [midway, sloped, below] {\footnotesize send()}
	(ether.north west);
  }

  \onslide<3>{
    \note<3> {
	- An interrupt isn't necessarily related to some user task.\\
    }
    \node [name=timer, right=5mm of ether, draw, rounded corners=1mm]
	{ timer };
    \draw [->] (timer.north) .. controls +(-1cm,2cm) and +(1cm,2cm) ..
	(timer.north);
  }

  \onslide<4>{
    \note<4> {
	- Let students think theirselves on possibility of such
	  upcall. Provide hints if needed.\\
    }
    \draw [color=red, ->] (bottom.north) to [out=140, in=270]
	node[midway, sloped, above] {is that possible?}
	(usertext.south west);
  }
\end{tikzpicture}
\end{frame}


\begin{frame}[fragile]
\frametitle{Adding a system call in FreeBSD}
\note<1> {
	- It is simple. Let's go through the steps.\\
	- Explain all fields.\\
}
\shellcmd{%
\# cd /usr/src/sys/kern\\
\# vi syscalls.master\\
}
\begin{beamercolorbox}[rounded=true,shadow=true,sep=0pt,ht=3.2cm]{editor}
\scriptsize \begin{verbatim}
541     AUE_ACCEPT      STD     { int accept4(int s, \
                                    struct sockaddr * __restrict name, \
                                    __socklen_t * __restrict anamelen, \
                                    int flags); }
542     AUE_PIPE        STD     { int pipe2(int *fildes, int flags); }
543     AUE_NULL        NOSTD   { int aio_mlock(struct aiocb *aiocbp); }   
544     AUE_NULL        STD     { int procctl(idtype_t idtype, id_t id, \
                                    int com, void *data); }
545     AUE_NULL        NOSTD   { int foobar(int foo, void *bar
\end{verbatim}
\end{beamercolorbox}
\shellcmd{%
\# make sysent\\
\# cd ../..\\
\# make buildkernel installkernel
}
\onslide<2->{
  \color{red}{
    I forgot to implement the sys\_foobar() function.
  }
}
\end{frame}


\begin{frame}
\frametitle{syscall ABI}
\note<1> {
	- What actually 'make sysent' does.\\
	- ABI is a convention on how arguments are passed between userland
	  and kernel.\\
	- A table of syscalls.\\
	- What if provide multiple ABIs? Yep, and that's the case.\\
	- Emphasize that Linux ABI isn't an emulator, but a compatibility
	  layer, very much like freebsd32 layer. The open(2) is special.\\
}
\emph{make sysent} generates:
\begin{itemize}
  \item /usr/include/sys/syscall.h (for userland)
  \item /usr/include/sys/sysproto.h (for kernel)
\end{itemize}

\onslide<2-> {
  There are many of them:
  \begin{itemize}
  \item sys/kern/syscalls.master
  \item sys/compat/freebsd32/syscalls.master
  \item sys/i386/linux/syscalls.master
  \item sys/amd64/linux32/syscalls.master
  \color{Agrey}{
  \item sys/compat/svr4/syscalls.master
  \item sys/i386/ibcs2/syscalls.master
  }
  \end{itemize}
}
\end{frame}


\begin{frame}[fragile]
\frametitle{Adding a system call in FreeBSD (dynamic way)}
\note {
}
\begin{itemize}
  \begin{item}Kernel side
    \begin{itemize}
      \item \Man{SYSCALL\_MODULE}{9}
      \item \Man{module}{9}
    \end{itemize}
  \end{item}
  \begin{item}Application side
    \begin{itemize}
      \item \Man{modstat}{2}
      \item \Man{syscall}{2}
    \end{itemize}
  \end{item}
\end{itemize}
\shellcmd{%
\# svn co http://svn.freebsd.org/base/user/glebius/course\\
\# cd course/02.entering\_kernel/syscall/module\\
\# vi foo\_syscall.c
}
\end{frame}

\end{document}
