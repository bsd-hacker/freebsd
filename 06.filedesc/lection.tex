\documentclass{beamer}

\usepackage[utf8]{inputenc}
\usepackage[russian]{babel}
\usepackage{tikz}
\usepackage{adjustbox}
\usepackage{url}
\usepackage{array}
\usepackage{xcolor}
\usepackage{listings}
\usepackage{verbatim}
\usepackage{ifthen}

\usetikzlibrary{positioning}
\usetikzlibrary{shapes}
\usetikzlibrary{decorations.pathmorphing}

\definecolor{yellowgreen}{HTML}{D0F000}
\definecolor{man}{HTML}{FF5E7C}
\definecolor{Agrey}{HTML}{AAAAAA}
\definecolor{Bgrey}{HTML}{BBBBBB}
\definecolor{Cgrey}{HTML}{CCCCCC}
\definecolor{Dgrey}{HTML}{DDDDDD}
\definecolor{Egrey}{HTML}{EEEEEE}

\definecolor{struct}{HTML}{999999}

% trace
\setbeamercolor{trace}{fg=black,bg=Cgrey}

% source
\setbeamercolor{source}{fg=black,bg=Cgrey}

% terminal
\setbeamercolor{terminal}{fg=white,bg=black}

% editor
\setbeamercolor{editor}{fg=black,bg=Cgrey}

% manual pages
\newcommand{\Man}[2]{
  \colorbox{white}{\color{man}#1(#2)}
}
\setbeamercolor{manref}{fg=black,bg=man}
\newcommand{\manlabel}[1]{
  \begin{beamercolorbox}[rounded=true,shadow=true,sep=0pt,colsep=0pt]{manref}
  \small{#1}
  \end{beamercolorbox}
}

% Source reference box
\setbeamercolor{srcref}{fg=black,bg=yellowgreen}
\newcommand{\srcref}[1]{
  \begin{beamercolorbox}[rounded=true,shadow=true,sep=0pt,colsep=0pt]{srcref}
  \tiny{Source code reference:}\footnotesize{ #1}
  \end{beamercolorbox}
}

% Shell command
\newcommand{\shellcmd}[1]{
  \begin{beamercolorbox}[rounded=true,shadow=true,sep=0pt,colsep=0pt]{terminal}
  \small{#1}
  \end{beamercolorbox}
}

% Source line
\newcommand{\srcline}[1]{
  \begin{beamercolorbox}[rounded=true,shadow=true,sep=0pt,colsep=0pt]{source}
  \small{#1}
  \end{beamercolorbox}
}

\tikzset { struct/.style = {
		draw, thick,
                rectangle split,
		rectangle split part fill={struct!50, white!50},
} }

%
% Footnotes for source/manual references.
%
\newcommand\ManRefSw{}
\newcommand\SrcRefSw{}
\makeatletter
\newcommand*\FootReferences[2]{
	\renewcommand\ManRefSw{#1}
	\renewcommand\SrcRefSw{#2}
	\beamer@calculateheadfoot
}
\makeatother
\setbeamertemplate{footline}{%
\ifthenelse{\NOT \equal{\ManRefSw}{}} {
	\begin{beamercolorbox}
	    [wd=\paperwidth,ht=2.25ex,dp=1ex,left]{manref}%
		\hspace*{1em}Manual page(s): \ManRefSw
	\end{beamercolorbox}
  }
  % else
  {}
\ifthenelse{\NOT \equal{\SrcRefSw}{}} {
	\begin{beamercolorbox}
	    [wd=\paperwidth,ht=2.25ex,dp=1ex,left]{srcref}%
		\hspace*{1em}Source code reference: \SrcRefSw
	\end{beamercolorbox}
  }
  % else
  {}
}


\title{File descriptors, capabilities, descriptor multiplexing}

\begin{document}

\begin{frame}
\titlepage
\end{frame}


\FootReferences{open(2), socket(2), dup(2), dup2(2), dup3(2), fork(2)}{}
\begin{frame}
\frametitle{File descriptors properties}
\onslide<1-> {
  Allocation:
  \srcline {%
	int fd;\\

	fd = open(path, flags, ...);\\
	fd = socket(domain, type, protocol);\\
	pipe(fds[]);
  }
}
\onslide<2-> {
  Usage:
  \srcline {%
	read(fd, buf, size);\\
	write(fd, buf, size);
  }
}
\onslide<3-> {
  Explicit duplication:
  \srcline {newfd = dup(fd);}
}
\onslide<4-> {
  Implicit duplication:
  \srcline {fork();}
}
\end{frame}


\FootReferences{unix(4)}{tools/regression/sockets/unix\_passfd}
\begin{frame}
\frametitle{File descriptors properties (non-standard)}
Passing descriptors via local socket:
\srcline {%
        msghdr.msg\_control = message;\\
        msghdr.msg\_controllen = sizeof(message);\\
        cmsghdr = (struct cmsghdr *)message;\\
        cmsghdr->cmsg\_len = CMSG\_LEN(sizeof(int));\\
        cmsghdr->cmsg\_level = SOL\_SOCKET;\\
        cmsghdr->cmsg\_type = SCM\_RIGHTS;\\
        *(int *)CMSG\_DATA(cmsghdr) = fd;\\
        sendmsg(sockfd, \&msghdr, 0);
}
Receiving:
\srcline {%
	recvmsg(sockfd, \&msghdr, 0);\\
	cmsghdr = CMSG\_FIRSTHDR(\&msghdr);\\
	fd = *(int *)CMSG\_DATA(cmsghdr);\\
}
\end{frame}


\FootReferences{}{sys/sys/file.h, sys/sys/filedesc.h}
\begin{frame}
\frametitle{File descriptors inside kernel}
\begin{figure}
\begin{tikzpicture}
  \node [name=file, struct, rectangle split parts=4] {
	\textbf{struct file}
	\nodepart{two} short f\_type
	\nodepart{three} void *f\_data
	\nodepart{four} struct fileops *f\_ops
  };
\onslide <2-> {
  \node [name=process, left=.4\paperwidth of file, yshift=.3\paperheight,
	 draw, circle] {
	user process
  };
  \node [name=fd, anchor=north, node distance=3mm,
	 below right=of process.center, draw, circle, inner sep=1pt] { fd };
}
\only<3> {
  \draw [->, decorate, decoration={snake, amplitude=1mm, segment length=1cm}]
	(fd) -- node [above, sloped] {???} (file.one west);
}
\only <4> {
  \node [name=filedesc, struct, rectangle split parts=2,
	 left=of file.north west, anchor=north east] {
	\textbf{struct filedesc}
	\nodepart{two} struct file fd\_ofiles[{\color{red}fd}]
  };
}
\onslide <5-> {
  \node [name=filedesc, struct, rectangle split parts=2,
	 left=of file.north west, anchor=north east] {
	\textbf{struct filedesc}
	\nodepart{two} struct \emph{filedescent} fd\_ofiles[{\color{red}fd}]
  };
}
\onslide <4-> {
  \node [name=proc, struct, rectangle split parts=2,
	 right=of process] {
	\textbf{struct proc}
	\nodepart{two} struct filedesc *p\_fd
  };
  \draw [->, decorate, decoration=snake] (process) -- (proc.one west);
  \draw [pointer] (proc.two west)
	 -- node [name=x,pos=1] {} +(-5mm,0) -- (filedesc.north -| x);
  \draw [pointer] (filedesc.two east) to [out=0, in=180] (file.one west);
}
\end{tikzpicture}
\end{figure}
\end{frame}


\FootReferences{capsicum(4), cap\_enter(2)}{}
\begin{frame}
\frametitle{Capability mode}
\end{frame}

\end{document}
