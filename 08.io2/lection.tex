\documentclass{beamer}

\usepackage[utf8]{inputenc}
\usepackage[russian]{babel}
\usepackage{tikz}
\usepackage{adjustbox}
\usepackage{url}
\usepackage{array}
\usepackage{xcolor}
\usepackage{listings}
\usepackage{verbatim}
\usepackage{ifthen}

\usetikzlibrary{positioning}
\usetikzlibrary{shapes}
\usetikzlibrary{decorations.pathmorphing}
\usetikzlibrary{decorations.text}

\definecolor{yellowgreen}{HTML}{D0F000}
\definecolor{man}{HTML}{FF5E7C}
\definecolor{Agrey}{HTML}{AAAAAA}
\definecolor{Bgrey}{HTML}{BBBBBB}
\definecolor{Cgrey}{HTML}{CCCCCC}
\definecolor{Dgrey}{HTML}{DDDDDD}
\definecolor{Egrey}{HTML}{EEEEEE}

\definecolor{struct}{HTML}{999999}

% trace
\setbeamercolor{trace}{fg=black,bg=Cgrey}

% source
\setbeamercolor{source}{fg=black,bg=Cgrey}

% terminal
\setbeamercolor{terminal}{fg=white,bg=black}

% editor
\setbeamercolor{editor}{fg=black,bg=Cgrey}

% manual pages
\newcommand{\Man}[2]{
  \colorbox{white}{\color{man}#1(#2)}
}
\setbeamercolor{manref}{fg=black,bg=man}
\newcommand{\manlabel}[1]{
  \begin{beamercolorbox}[rounded=true,shadow=true,sep=0pt,colsep=0pt]{manref}
  \small{#1}
  \end{beamercolorbox}
}

% Source reference box
\setbeamercolor{srcref}{fg=black,bg=yellowgreen}
\newcommand{\srcref}[1]{
  \begin{beamercolorbox}[rounded=true,shadow=true,sep=0pt,colsep=0pt]{srcref}
  \tiny{Source code reference:}\footnotesize{ #1}
  \end{beamercolorbox}
}

% Shell command
\newcommand{\shellcmd}[1]{
  \begin{beamercolorbox}[rounded=true,shadow=true,sep=0pt,colsep=0pt]{terminal}
  \small{#1}
  \end{beamercolorbox}
}

% Source line
\newcommand{\srcline}[1]{
  \begin{beamercolorbox}[rounded=true,shadow=true,sep=0pt,colsep=0pt]{source}
  \small{#1}
  \end{beamercolorbox}
}

\tikzset { struct/.style = {
		draw, thick,
                rectangle split,
		rectangle split part fill={struct!50, white!50},
} }

%
% Footnotes for source/manual references.
%
\newcommand\ManRefSw{}
\newcommand\SrcRefSw{}
\makeatletter
\newcommand*\FootReferences[2]{
	\renewcommand\ManRefSw{#1}
	\renewcommand\SrcRefSw{#2}
	\beamer@calculateheadfoot
}
\makeatother
\setbeamertemplate{footline}{%
\ifthenelse{\NOT \equal{\ManRefSw}{}} {
	\begin{beamercolorbox}
	    [wd=\paperwidth,ht=2.25ex,dp=1ex,left]{manref}%
		\hspace*{1em}Manual page(s): \ManRefSw
	\end{beamercolorbox}
  }
  % else
  {}
\ifthenelse{\NOT \equal{\SrcRefSw}{}} {
	\begin{beamercolorbox}
	    [wd=\paperwidth,ht=2.25ex,dp=1ex,left]{srcref}%
		\hspace*{1em}Source code reference: \SrcRefSw
	\end{beamercolorbox}
  }
  % else
  {}
}


\title{Input/Output system: block I/O and GEOM}

\begin{document}

\begin{frame}
\titlepage
\end{frame}

\begin{frame}
\frametitle{I/O below VFS}
Filesystem code posts I/O request:
\only <1> {
request passes down through the I/O to the hardware layer.
}
\only <2> {
driver returns, geom returns, filesystem code sleeps.
}
\only <3> {
I/O comletion interrupt triggers, wakeups thread waiting on bio,
both threads return.
}
\begin{figure}
\small\begin{tikzpicture}[
	every node/.style={node distance=2mm}
  ]
  \node [name=user, draw, rounded corners] { write(2) };
  \node [name=sys, below=of user] { sys\_write() };
  \draw [->] (user) -- (sys);
  \node [name=vn, below=of sys] { vn\_write() };
  \draw [->] (sys) -- (vn);
  \node [name=vop, below=of vn] { VOP\_WRITE\_APV() };
  \draw [->] (vn) -- (vop);
  \node [name=fs, below=of vop] { fs specific code: ufs, zfs, ext2fs, ... };
  \draw [->] (vop) -- (fs);
\onslide <1> {
  \node [name=geom, below=of fs] { geom classes };
  \draw [->] (fs) -- (geom);
  \node [name=disk, below=of geom] { g\_disk\_start(bio) };
  \draw [->] (geom) -- (disk);
  \node [name=driver, below=of disk] { disk strategy(bio) };
  \draw [->] (disk) -- (driver);
  \node [name=hw, below=of driver] { I/O posted to hardware };
  \draw [->] (driver) -- (hw);
}

  \node [name=uk1, below left=1mm and .2\paperwidth of user] {};
  \node [name=uk2, below right=1mm and .4\paperwidth of user] {};
  \draw (uk1) --
	node [above, pos=.9] { userland }
	node [below, pos=.9] { kernel } (uk2);

  \node [name=kd1, below left=1mm and .2\paperwidth of disk] {};
  \node [name=kd2, below right=1mm and .4\paperwidth of disk] {};
  \draw (kd1) --
	node [below, pos=.9, name=drivertext] { driver } (kd2);

\onslide <2-3> {
  \node [name=sleep, below=of fs] { sleep(\&bio) };
  \draw [->] (fs) -- (sleep);
}

\onslide <3> {
  \node [name=intr, below left=of drivertext, draw, rounded corners]
	{ interrupt };
  \node [name=dintr, above=of intr] { driver code };
  \draw [->] (intr) -- (dintr);
  \node [name=biodone, above=of dintr] { biodone() };
  \draw [->] (dintr) -- (biodone);
  \node [name=wakeup, above=of biodone] { wakeup(bio) };
  \draw [->] (biodone) -- (wakeup);
  \draw [->, color=red] (wakeup.south west)
	to [out=225, in=325] (sleep.south east);
}
\end{tikzpicture}
\end{figure}
\end{frame}


\begin{frame}
\frametitle<1-2>{Block storage API: basic block storage}
\frametitle<3>{Block storage API: removable block storage}
\frametitle<4>{Block storage API: write-caching block storage}
\frametitle<5>{Block storage API: thin-provisioned block storage}
\frametitle<6>{Block storage API: additional attributes}
\begin{columns}
  \begin{column}{.4\paperwidth}
  \begin{itemize}
\onslide<3-> {
  \item{Media lock/notify}
}
  \item{Data operations:
    \begin{itemize}
      \item{Read}
      \item{Write}
\onslide <4-> {
      \item{Cache flush}
}
\onslide <5-> {
      \item{Unmap/Trim}
}
    \end{itemize}
  }
  \item{Properties:
    \begin{itemize}
      \item{Block size}
      \item{Capacity}
\onslide <6-> {
      \item {C/H/S, physical sector size, \ldots}
}
    \end{itemize}
  }
  \end{itemize}
  \end{column}
  \begin{column}{.4\paperwidth}
  \begin{itemize}
\onslide<3-> {
  \item{access(), spoiled()}
}
\onslide <2-> {
  \item{strategy(struct bio *)
    \begin{itemize}
      \item{BIO\_READ}
      \item{BIO\_WRITE}
\onslide <4-> {
      \item{BIO\_FLUSH}
}
\onslide <5-> {
      \item{BIO\_DELETE}
}
    \end{itemize}
  }
    \begin{itemize}
      \item{sector size}
      \item{media size}
\onslide <6-> {
      \item {stripe size, stripe offset, BIO\_GETATTR }
}
    \end{itemize}
}
  \end{itemize}
  \end{column}
\end{columns}
\end{frame}


\FootReferences{disk(9)}{sys/geom/geom\_disk.h}
\begin{frame}[fragile]
\frametitle{The disk(9) API}
\small\begin{verbatim}
struct disk {
        disk_open_t             *d_open;
        disk_close_t            *d_close;
        disk_strategy_t         *d_strategy;
        disk_ioctl_t            *d_ioctl;
        dumper_t                *d_dump;
        disk_getattr_t          *d_getattr;
        disk_gone_t             *d_gone;

        u_int                   d_sectorsize;
        u_int                   d_maxsize;
        off_t                   d_mediasize;

        u_int                   d_stripeoffset;
        u_int                   d_stripesize;
}
\end{verbatim}
\end{frame}


\FootReferences{disk(9)}{sys/geom/geom\_disk.h}
\begin{frame}
\frametitle{The disk(9) API}
Brief history:
\begin{itemize}
  \item{FreeBSD 3: d\_strategy(), d\_open(), d\_close() is provided by cdevsw}
  \item{FreeBSD 4: disk(9) introduced}
  \item{FreeBSD 5: disk(9) is a GEOM class}
\end{itemize}
\end{frame}


\FootReferences{geom(9)}{}
\begin{frame}
\frametitle{And what is GEOM?}
Assume DOS partitioning scheme on disk ada0.
\onslide <2-> {
  Now assume BSD partitioning scheme on ``disk'' ada0s2.
}
  \begin{figure}
  \begin{tikzpicture}[
	every node/.style={ draw, rounded corners, text centered,
			    text width = 20ex, node distance = 3mm }
    ]
    \node [name=ada0] { ada0\\ mediasize = 100 Gb };
    \node [name=ada0s1, above left=of ada0.north]
	{ ada0s1\\ mediasize = 50 Gb };
    \node [name=ada0s2, above right=of ada0.north]
	{ ada0s2\\ mediasize = 50 Gb };
    \draw [->] (ada0) -- (ada0s1);
    \draw [->] (ada0) -- (ada0s2);
\onslide <2-> {
    \node [name=ada0s2a, text width = 10ex, above left=of ada0s2.north]
	{ ada0s2a\\ 10 Gb };
    \node [name=ada0s2b, text width = 10ex, above right=of ada0s2.north]
	{ ada0s2b\\ 40 Gb };
    \draw [->] (ada0s2) -- (ada0s2a);
    \draw [->] (ada0s2) -- (ada0s2b);
}
  \end{tikzpicture}
  \end{figure}
\end{frame}


\FootReferences{geom(9)}{}
\begin{frame}
\frametitle{And what is GEOM?}
\only <1> {
Another example: a stripe of disks aka RAID0
}
\only <2> {
... or a disk mirror aka RAID1
}
\only <3> {
Who said that mirror can be built only on bare disks?
}
  \begin{figure}
  \begin{tikzpicture}[
	every node/.style={ draw, rounded corners, text centered,
			    text width = 20ex, node distance = 3mm }
    ]
\only <1> {
    \node [name=top] { mirror\\ mediasize = 200 Gb };
}
\only <2-> {
    \node [name=top] { stripe\\ mediasize = 100 Gb };
}
\onslide <1-2> {
    \node [name=ada0, below left=of top.south]
	{ ada0\\ mediasize = 100 Gb };
    \node [name=ada1, below right=of top.south]
	{ ada1\\ mediasize = 100 Gb };
}
\onslide <3-> {
    \node [name=ada0, below left=of top.south]
	{ ada0{\color{red}a}\\ mediasize = 100 Gb };
    \node [name=ada1, below right=of top.south]
	{ ada1{\color{red}a}\\ mediasize = 100 Gb };
    \node [name=ada00, below=of ada0] { ada0 };
    \node [name=ada10, below=of ada1] { ada1 };
    \draw [->] (ada00) -- (ada0);
    \draw [->] (ada10) -- (ada1);
}
    \draw [->] (ada0) -- (top);
    \draw [->] (ada1) -- (top);
  \end{tikzpicture}
  \end{figure}
\end{frame}


\FootReferences{geom(9)}{}
\begin{frame}
\frametitle{GEOM — modular disk I/O request transformation framework}
\begin{itemize}
  \item {
    \textbf{class} - a particular transformation: partitioning, RAIDs,
	encryption. About 40 of them.
  }
\onslide <2-> {
  \item {
    \textbf{geom} - instance of a single class. A node in GEOM graph.
    Multiple \textbf{geom} instances of same class can exist.
  }
}
\onslide <3-> {
  \item {
    \textbf{provider} - a disk-like thing that a \textbf{geom} provides
    to devfs, or to other \textbf{geom}.
  }
}
\onslide <4-> {
  \item {
    \textbf{consumer} - an interface of a \textbf{geom}, that attaches to
    underlying \textbf{provider}.
  }
}
\end{itemize}
\end{frame}


\FootReferences{geom(9)}{}
\begin{frame}
\frametitle{provider and consumer}
\begin{figure}
\begin{tikzpicture}[
	every node/.style={ rounded corners, text centered }
  ]
  \node [name=up, draw] { ada0a };
  \node [name=down, draw, below=3cm of up] { ada0 };
  \draw [->, thick, postaction = {
		decorate, decoration = {
		  text along path, text align = center,
		  text = { consumer of }
		}}]
	(up.south east) to [out=315, in=45] (down.north east);
  \draw [->, thick, postaction = {
                decorate, decoration = {
                  text along path, text align = center,
                  text = { provider for }
                }}]
	(down.north west) to [out=135, in=225] (up.south west);
\end{tikzpicture}
\end{figure}
\end{frame}


\FootReferences{geom(9)}{sys/geom/geom\_subr.c}
\begin{frame}
\frametitle{GEOM graph is acyclic}
\begin{figure} 
\begin{tikzpicture}[
	every node/.style={ draw, rounded corners, text centered,
			    text width = 10ex, node distance = 3mm }
    ]
  \node [name=top, text width = 40ex]
	{ mirror\\ rank = max(rank of consumers) = 2 };
  \node [name=ada0, below left=of top.south]
	{ ada0\\ rank = 1 };
  \node [name=ada1, below right=of top.south]
	{ ada1\\ rank = 1 };
  \draw [->] (ada0) -- (top);
  \draw [->] (ada1) -- (top);
\end{tikzpicture}
\end{figure}
\end{frame}


\FootReferences{geom(9)}{sys/geom/geom\_subr.c}
\begin{frame}
\frametitle{GEOM topology management}
\begin{itemize}
  \item {
    \textbf{configuration} - manual request for a given class to instantiate
    itself, with certain parameters.
  }
\end{itemize}
\end{frame}

\end{document}
