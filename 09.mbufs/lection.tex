\documentclass{beamer}

\usepackage[utf8]{inputenc}
\usepackage[russian]{babel}
\usepackage{tikz}
\usepackage{adjustbox}
\usepackage{url}
\usepackage{array}
\usepackage{xcolor}
\usepackage{listings}
\usepackage{verbatim}
\usepackage{ifthen}

\usetikzlibrary{chains}
\usetikzlibrary{scopes}
\usetikzlibrary{positioning}
\usetikzlibrary{shapes}
\usetikzlibrary{patterns}

\definecolor{yellowgreen}{HTML}{D0F000}
\definecolor{man}{HTML}{FF5E7C}
\definecolor{Agrey}{HTML}{AAAAAA}
\definecolor{Bgrey}{HTML}{BBBBBB}
\definecolor{Cgrey}{HTML}{CCCCCC}
\definecolor{Dgrey}{HTML}{DDDDDD}
\definecolor{Egrey}{HTML}{EEEEEE}

\definecolor{struct}{HTML}{999999}

% trace
\setbeamercolor{trace}{fg=black,bg=Cgrey}

% source
\setbeamercolor{source}{fg=black,bg=Cgrey}

% terminal
\setbeamercolor{terminal}{fg=white,bg=black}

% editor
\setbeamercolor{editor}{fg=black,bg=Cgrey}

% manual pages
\newcommand{\Man}[2]{
  \colorbox{white}{\color{man}#1(#2)}
}
\setbeamercolor{manref}{fg=black,bg=man}
\newcommand{\manlabel}[1]{
  \begin{beamercolorbox}[rounded=true,shadow=true,sep=0pt,colsep=0pt]{manref}
  \small{#1}
  \end{beamercolorbox}
}

% Source reference box
\setbeamercolor{srcref}{fg=black,bg=yellowgreen}
\newcommand{\srcref}[1]{
  \begin{beamercolorbox}[rounded=true,shadow=true,sep=0pt,colsep=0pt]{srcref}
  \tiny{Source code reference:}\footnotesize{ #1}
  \end{beamercolorbox}
}

% Shell command
\newcommand{\shellcmd}[1]{
  \begin{beamercolorbox}[rounded=true,shadow=true,sep=0pt,colsep=0pt]{terminal}
  \small{#1}
  \end{beamercolorbox}
}

% Source line
\newcommand{\srcline}[1]{
  \begin{beamercolorbox}[rounded=true,shadow=true,sep=0pt,colsep=0pt]{source}
  \small{#1}
  \end{beamercolorbox}
}

\tikzset { struct/.style = {
		draw, thick,
                rectangle split,
		rectangle split part fill={struct!50, white!50},
} }

%
% Footnotes for source/manual references.
%
\newcommand\ManRefSw{}
\newcommand\SrcRefSw{}
\makeatletter
\newcommand*\FootReferences[2]{
	\renewcommand\ManRefSw{#1}
	\renewcommand\SrcRefSw{#2}
	\beamer@calculateheadfoot
}
\makeatother
\setbeamertemplate{footline}{%
\ifthenelse{\NOT \equal{\ManRefSw}{}} {
	\begin{beamercolorbox}
	    [wd=\paperwidth,ht=2.25ex,dp=1ex,left]{manref}%
		\hspace*{1em}Manual page(s): \ManRefSw
	\end{beamercolorbox}
  }
  % else
  {}
\ifthenelse{\NOT \equal{\SrcRefSw}{}} {
	\begin{beamercolorbox}
	    [wd=\paperwidth,ht=2.25ex,dp=1ex,left]{srcref}%
		\hspace*{1em}Source code reference: \SrcRefSw
	\end{beamercolorbox}
  }
  % else
  {}
}


\title{Network Input/Output: mbufs}

\begin{document}

\begin{frame}
\titlepage
\end{frame}

\begin{frame}
\frametitle{Basic view of network I/O}
\begin{figure}
\begin{tikzpicture}[thick]
  \node [name=sockbuf, draw, rounded corners, minimum width=.4\paperwidth,
	 minimum height=2em]
	{ socket send/receive buffers };
  \node [name=driver, draw, rounded corners, minimum width=.4\paperwidth,
	 minimum height=2em, below=.3\paperheight of sockbuf]
	{ NIC driver RX/TX rings };
  \node [name=write, above left=of sockbuf.center] { write(2) };
  \node [name=read, above right=of sockbuf.center] { read(2) };

  \node [name=uk1, above left=2mm and .2\paperwidth of sockbuf] {};
  \node [name=uk2, above right=2mm and .2\paperwidth of sockbuf] {};
  \draw (uk1) --
	node [above, pos=.9] { userland }
	node [below, pos=.9] { kernel } (uk2);

  \draw [->] (write) -- (write |- sockbuf.north);
  \draw [<-] (read) -- (read |- sockbuf.north);

  \node (mark1) at (node cs:name=sockbuf, angle=195) {};
  \node (mark2) at (node cs:name=sockbuf, angle=345) {};
  \draw [->] (mark1.center) -- (mark1 |- driver.north);
  \draw [<-] (mark2.center) -- (mark2 |- driver.north);
\end{tikzpicture}
\end{figure}
\end{frame}


\begin{frame}
\frametitle{I/O metadata: protocol headers}
  \begin{itemize}
    \item{Unlike disk I/O, metadata is embedded into data}
    \item{Metadata variates at different levels of I/O:
	\begin{itemize}
	  \item{Socket buffers are free of metadata}
	  \item{NIC driver rings has protocol headers of all network levels}
	\end{itemize}
    }
  \end{itemize}
\end{frame}


\begin{frame}
\frametitle{I/O sizing}
  \begin{itemize}
    \item {Size of protocol headers is unpredictable,
	 variates around 10 - 100 bytes}
    \item {Payload size:
	\begin{itemize}
	  \item{TCP acknowledgement only packet has 0 sized payload}
	  \item{Typical Internet MTU 1500 bytes (minus protocol headers)}
	  \item{There is also Jumbo MTU of 9000 bytes}
	\end{itemize}
    }
    \item {Typical memory page size is 4096 bytes}
    \item {Driver alignment restrictions}
  \end{itemize}
\end{frame}


\begin{frame}
\frametitle{Demands for network I/O buffer}
\begin{itemize}
  \item{Variable memory size}
  \item{Prependable and extendable}
  \item{Queueable}
\end{itemize}
\end{frame}


\begin{frame}
\frametitle{The mbuf}
\begin{figure}
\begin{tikzpicture}[
  start chain=going right, node distance=0mm,
  mpart/.style= {
	on chain,
	minimum height=3em,
	draw, thick,
	align=center
  }
]
\onslide <1-> {
  \node [mpart, name=mhdr] { m\_hdr };
  \node [mpart, name=empty1, text width=.05\paperwidth] { };
  \node [mpart, name=proto1, pattern=checkerboard light gray, text width=8ex]
	{ protocol headers };
  \node [mpart, name=payload, fill=gray, shading=axis, shading angle=45,
	 text width=.10\paperwidth]
	{ payload };
  \node [mpart, name=proto2, pattern=checkerboard light gray, text width=8ex]
	{ protocol footers };
  \node [mpart, name=empty2, text width=.15\paperwidth] { };
}
\onslide <2-> {
  \node [name=struct, below right=2em and 2em of mhdr.south east,
	 anchor=north east, 
	 struct, rectangle split parts=6] {
	\textbf{struct m\_hdr}
	\nodepart{two}char *m\_data
	\nodepart{three}int32\_t m\_len
	\nodepart{four}uint32\_t m\_flags
	\nodepart{five}struct mbuf *m\_next
	\nodepart{six}struct mbuf *m\_nextpkt
  };
  \draw (mhdr.south east) -- (struct.north east);
  \draw (mhdr.south west) -- (struct.north west);
  \draw [->, thick] (struct.two east) to [out=0, in=270] (proto1.south west);

  \node [name=mark1, below=2mm of proto1.south west] {};
  \node [name=mark2, below=2mm of proto2.south east] {};
  \draw [<->] (mark1.center) -- node (mark3) {} (mark2.center);
  \draw [->, thick, rounded corners] (struct.three east) -| (mark3.center);

  \node [name=mark4, above=2mm of mhdr.north west] {};
  \node [name=mark5, above=2mm of empty2.north east] {};
  \draw [<->] (mark4.center) --
	node [above] { MSIZE (256 bytes) } (mark5.center);
}
\end{tikzpicture}
\end{figure}
\end{frame}


\begin{frame}
\frametitle{Mbuf chain}
\begin{figure}
\begin{tikzpicture}[
  every node/.style= {
	node distance=0mm,
	minimum height=3em,
	thick, align=center
  }
]
{ [start chain=1 going right]
  \node [draw, on chain=1, name=mhdr] { m\_hdr };
  \node [draw, on chain=1, name=payload,
	 fill=gray, shading=axis, shading angle=45,
	 minimum width=.2\paperwidth]
	{ payload };
  \node [draw, on chain=1, name=empty2, minimum width=.4\paperwidth] { };
}
{ [start chain=2 going right]
  \node [name=mhdr2, below=2 of mhdr, draw, on chain=2] { m\_hdr };
  \node [draw, on chain=2, name=payload,
	 fill=gray, shading=axis, shading angle=45,
	 minimum width=.6\paperwidth]
	{ payload continues};
}
  \draw [->, thick] (mhdr.south) --
	node [above, rotate=90, minimum height=1em] {m\_next} (mhdr2.north);
\end{tikzpicture}
\end{figure}
\end{frame}


\begin{frame}
\frametitle{The packet header mbuf}
\begin{figure}
\begin{tikzpicture}[
  start chain=going right, node distance=0mm,
  mpart/.style= {
	on chain,
	minimum height=3em,
	draw, thick,
	align=center
  }
]
  \node [mpart, name=mhdr] { m\_hdr };
  \node [mpart, name=pkthdr] { pkthdr };
  \node [mpart, name=empty, text width=.3\paperwidth] { };

  \node [name=mhdrstr, below left=1em and 1em of mhdr.south east,
	 anchor=north east, 
	 struct, rectangle split parts=3] {
	\textbf{struct m\_hdr}
	\nodepart{two}char *m\_data
	\nodepart{three}uint32\_t m\_flags
  };
  \node [right=of mhdrstr.three east, anchor=west, color=red] { \& M\_PKTHDR };
  \draw (mhdr.south west) -- (mhdrstr.north west);
  \draw (mhdr.south east) -- (mhdrstr.north east);

  \node [name=pkthdrstr, below=.2\paperwidth of empty.south east,
	 anchor=north east, struct, rectangle split parts=4] {
	\textbf{struct pkthdr}
	\nodepart{two}struct ifnet *rcvif
	\nodepart{three}int32\_t len
	\nodepart{four}SLIST\_HEAD(, m\_tag) tags
  };
  \draw (pkthdr.south west) -- (pkthdrstr.north west);
  \draw (pkthdr.south east) -- (pkthdrstr.north east);
\end{tikzpicture}
\end{figure}
\end{frame}


\begin{frame}
\frametitle{The M\_EXT mbuf and mbuf cluster}
\begin{figure}
\begin{tikzpicture}[
  start chain=going right, node distance=0mm,
  mpart/.style= {
	on chain,
	minimum height=3em,
	draw, thick,
	align=center
  }
]
\onslide <1-> {
  \node [mpart, name=mhdr] { m\_hdr };
  \node [mpart, name=pkthdr] { \textit{pkthdr} };
  \node [mpart, name=mext] { m\_ext };
  \node [mpart, name=empty, text width=.3\paperwidth] { };

  \node [name=mhdrstr, below left=1em and 1em of mhdr.south east,
	 anchor=north east, 
	 struct, rectangle split parts=3] {
	\textbf{struct m\_hdr}
	\nodepart{two}char *m\_data
	\nodepart{three}uint32\_t m\_flags
  };
  \node [right=of mhdrstr.three east, anchor=west, color=red] { \& M\_EXT };
  \draw (mhdr.south west) -- (mhdrstr.north west);
  \draw (mhdr.south east) -- (mhdrstr.north east);
}
\onslide <2-> {
  \node [name=mextstr, below=1em of empty.south east, anchor=north east,
	 struct, rectangle split parts=6] {
	\textbf{struct m\_ext}
	\nodepart{two}u\_int *ref\_cnt
	\nodepart{three}char *ext\_buf
	\nodepart{four}uint32\_t ext\_size
	\nodepart{five}uint32\_t ext\_flags
	\nodepart{six}int (*ext\_free)(...)
  };
  \draw (mext.south west) -- (mextstr.north west);
  \draw (mext.south east) -- (mextstr.north east);

  \node [name=cluster, draw, below=1em of mextstr.south east, anchor=north east,
	 minimum height=3em, text width=.8\paperwidth, align=center]
	{ \textit{cluster} (2048 bytes) };

  \draw [->,thick] (mextstr.three west)
	.. controls +(-3,-3) and +(0,2) .. (cluster.north west);
  \draw [->,thick] (mhdrstr.two east)
	.. controls +(2,-2) and +(0,2) .. (node cs:name=cluster, angle=170);
}
\end{tikzpicture}
\end{figure}
\end{frame}


\FootReferences{}{sys/kern/kern\_mbuf.c}
\begin{frame}
\frametitle{Allocation of mbufs}
  \begin{itemize}
    \item{UMA zones:
	\begin{itemize}
	  \item{mbuf (256 bytes) zone}
	  \item{cluster (2048 bytes) zone}
\onslide <2-> {
	  \item{``packet'' zone}
}
\onslide <3-> {
	  \item{PAGE\_SIZEd zone}
	  \item{Jumbo-sized zone}
}
\onslide <4-> {
	  \item{refcount zone (sizeof(u\_int))}
}
	\end{itemize}
    }
\onslide <5-> {
    \item {Statistics can be analyzed with vmstat(8), netstat(1)}
}
\onslide <6-> {
    \item {Usually allocation happens:
	\begin{itemize}
	  \item{Driver fills its RX ring}
	  \item{User data is copied into socket buffer on write(2)}
	\end{itemize}
    }
}
\onslide <7-> {
    \item {Layers pass ownership of mbufs to each other}
}
  \end{itemize}
\end{frame}


\FootReferences{mbuf(9)}{sys/sys/mbuf.h, sys/kern/kern\_mbuf.c}
\begin{frame}
\frametitle{Allocation APIs}
  \begin{itemize}
    \item{Explicit allocation:
	\begin{itemize}
    	  \item{m\_get(how, type) - allocate 256 byte mbuf}
    	  \item{m\_gethdr(how, type) - allocate M\_PKTHDR mbuf}
    	  \item{m\_getcl(how, type, flags) - allocate ``packet'': mbuf+cluster}
\onslide <2-> {
    	  \item{m\_get2(size, how, type, flags) - allocate minimum sized buffer}
    	  \item{m\_getm(size, how, type, flags) - allocate chain of buffers}
}
	\end{itemize}
    }
\onslide <3-> {
    \item{Implicit allocation:
	\begin{itemize}
	  \item{m\_uiotombuf(uio, ...) - copy uio(9) to allocated chain}
	  \item{m\_devget(char *, ...) - copy buffer to allocated chain}
	  \item{m\_copym(mbuf, ...) - copy mbuf(9) chain to new one}
	\end{itemize}
    }
}
\onslide <4-> {
    \item{Freeing:
	\begin{itemize}
	  \item{m\_free(m) - free this mbuf and return its m\_next}
	  \item{m\_freem(m) - free entire chain}
	\end{itemize}
    }
}
  \end{itemize}
\end{frame}


\FootReferences{mbuf(9)}{sys/sys/mbuf.h, sys/kern/uipc\_mbuf.c}
\begin{frame}
\frametitle{Dereferencing, prepending and aligning mbufs}
  \begin{itemize}
    \item{Dereferencing:
	\begin{itemize}
	  \item{protoheader = mtod(m, cast) -- cast m\_data to a header}
	  \item{Must check m\_len!}
	  \item{m = m\_prepend(m, len) --
		guarantee m\_data contiguity of len bytes}
	\end{itemize}
    }
\onslide <2-> {
    \item{Prepending:
	\begin{itemize}
	  \item{M\_PREPEND(m, len, how) -- macro to prepend len before m\_data}
	  \item{m\_prepend(m, len, how) -- hard part of M\_PREPEND}
	\end{itemize}
    }
}
\onslide <3-> {
    \item{Aligning:
	\begin{itemize}
	  \item{M\_ALIGN(m, len), MH\_ALIGN(m, len) -- set m\_data to len bytes
		at end of data area}
	  \item{m\_align(m, len) -- same for any type of mbuf}
	\end{itemize}
    }
}
  \end{itemize}
\end{frame}


\FootReferences{}{sys/netinet/ip\_input.c}
\begin{frame}[fragile]
\frametitle{Example: dereferencing mbufs}
\begin{beamercolorbox}[rounded=true,shadow=true]{source}
\lstset{language=C}
\scriptsize\begin{lstlisting}
void
ip_input(struct mbuf *m)
{
        struct ip *ip;

        if (m->m_pkthdr.len < sizeof(struct ip))
                goto tooshort;

        if (m->m_len < sizeof (struct ip) &&
            (m = m_pullup(m, sizeof (struct ip))) == NULL) {
                IPSTAT_INC(ips_toosmall);
                return;
        }
        ip = mtod(m, struct ip *);

        if (ip->ip_v != IPVERSION) {
                IPSTAT_INC(ips_badvers);
                goto bad;
        }
\end{lstlisting}
\end{beamercolorbox}
\end{frame}


\FootReferences{}{sys/net/if\_ethersubr.c}
\begin{frame}[fragile]
\frametitle{Example: prepending header}
\begin{beamercolorbox}[rounded=true,shadow=true]{source}
\lstset{language=C}
\scriptsize\begin{lstlisting}
int
ether_output(struct ifnet *ifp, struct mbuf *m,
        const struct sockaddr *dst, struct route *ro)
{
        struct ether_header *eh;

        M_PREPEND(m, ETHER_HDR_LEN, M_NOWAIT);
        if (m == NULL)
                senderr(ENOBUFS);
        eh = mtod(m, struct ether_header *);
        memcpy(&eh->ether_type, &type,
            sizeof(eh->ether_type));
        memcpy(eh->ether_dhost, edst,
            sizeof(eh->ether_dhost));
        memcpy(eh->ether_shost, IF_LLADDR(ifp),
            sizeof(eh->ether_shost));
\end{lstlisting}
\end{beamercolorbox}
\end{frame}


\end{document}
